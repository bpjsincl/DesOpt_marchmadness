\documentclass[12pt]{article}
\usepackage{sydewkrpt}
\usepackage{longtable}
\usepackage{array}
\usepackage{ragged2e}
\usepackage{amsmath}
\usepackage{amssymb}
\usepackage{siunitx}
\DeclareMathOperator*{\argmin}{\arg\!\min}
\DeclareMathOperator*{\argmax}{\arg\!\max}
\newcolumntype{P}[1]{>{\RaggedRight\hspace{0pt}}p{#1}}

%%%%%%%%%%%%%%%%%%%%%%%%%%%%
%%%    Begin Document    %%%
%%%%%%%%%%%%%%%%%%%%%%%%%%%%
\begin{document}
\pagenumbering{roman}

\waterlootitle{SYDE 531: Final Project Report}{
  Betting Strategies for an NCAA March Madness Tournament Bracket\\
}{
  Brian Sinclair -- 20346309\\
  Riley Donelson -- 20342815\\
  }

\dotableofcontents

\newpage
\doublespacing
\pagenumbering{arabic}
\section{Introduction}
\setlength{\parindent}{1cm}
\subsection{Background}

\subsection{Problem Definition}


\newpage
\section{Optimization Algorithms}
\subsection{Mean-Variance Approach to Portfolio Optimization}
\subsubsection{Approach Background}
As previously mentioned the goal of the project is to maximize the return on investment when betting on individual games in the yearly NCAA March Madness Tournament.
It was recognized early on that gambling in sports follows a similar behaviour to that of portfolio optimization between two assets. 
Any one game can be thought of as a portfolio where each asset is team A, or team B in the match-up.
The objective of any one game (as a portfolio) is to maximize return on investment when given a budget to spend on that game.
For the purposes of the tournament, instead of the portfolio being an individual game, it becomes an individual round (ie. Portfolio's 1 through 6 are Rounds 1 through 6).
The portfolio then changes from being constructed of assets as teams, to assets as each game.
In other words, one game is one asset in the portfolio (Round).

The bettor, or owner of this newly defined portfolio

\subsection{Dynamic Programming}

\section{Results and Analysis}
\subsection{Sample Inputs, Outputs, and Detailed Solution}


\subsection{Sensitivity Analysis}


\subsection{Inferences on Problem Solutions}

\newpage
\section{Conclusions}


\newpage
\section{Appendix A: Full Implementation}


\end{document}