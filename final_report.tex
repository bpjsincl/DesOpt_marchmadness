\documentclass[12pt]{article}
\usepackage{sydewkrpt}
\usepackage{longtable}
\usepackage{array}
\usepackage{ragged2e}
\usepackage{amsmath}
\usepackage{amssymb}
\usepackage{siunitx}
\DeclareMathOperator*{\argmin}{\arg\!\min}
\DeclareMathOperator*{\argmax}{\arg\!\max}
\newcolumntype{P}[1]{>{\RaggedRight\hspace{0pt}}p{#1}}

%%%%%%%%%%%%%%%%%%%%%%%%%%%%
%%%    Begin Document    %%%
%%%%%%%%%%%%%%%%%%%%%%%%%%%%
\begin{document}
\pagenumbering{roman}

\waterlootitle{SYDE 531: Final Project Report}{
  Betting Strategies for an NCAA March Madness Tournament Bracket\\
}{
  Brian Sinclair -- 20346309\\
  Riley Donelson -- 20342815\\
  }

\dotableofcontents

\newpage
\doublespacing
\pagenumbering{arabic}
\section{Introduction}
\setlength{\parindent}{1cm}
\subsection{Background}
This report details the design optimization of betting strategies for the NCAA Men's Division 1 Basketball Championship, commonly referred to as the ``March Madness'' tournament.
This tournament is a yearly event comprising of the top 64 teams in US College Basketball \cite{doi:10.1080/00031305.1996.10473540}.
It is structured in a classical tournament bracket format, beginning with 32 games (64 teams) and ending with 1 final championship game.
The losing team from each game is eliminated from the tournament at each round.

The attempt at predicting and/or betting on the games in this tournament is a common activity for many fans and followers of the sport.
In order to provide more informed decisions, the design optimization of betting strategies under probabilistic uncertainty was conducted using a modern portfolio theory mean-variance approach, combined with dynamic programming.
The formulation of each optimization algorithm is provided, as well as the results and analysis of the solution.
Discussions on the results, sensitivity of the system, and conclusions are also provided. 

\subsection{Problem Definition}


\newpage
\section{Optimization Algorithms}
\subsection{Mean-Variance Approach to Portfolio Optimization}

\subsection{Dynamic Programming}
The method of discrete dynamic programming was employed in this solution to take the values provided from the mean-variance approach to portfolio optimization, and select the best ``path'' through the tournament where bets should be placed to maximize return.
The betting possibilities are discretized into 5 states, wherein one may utilize 100\%, 75\%, 50\%, 25\%, or 0\% of their budget on a round of the tournament.

Using the discrete dynamic programming methodology outlined in class, the problem was divided into stages, with each stage representing a round of the tournament.
This method depends on Bellman's principle of optimality, stated as follows:
\begin{quote}
\emph{``The optimal set of decisions in a multistage process has the property that whatever the initial stage, state and decisions are, the remaining decisions, given the current stage, must form a set of optimal decisions for the remaining problem.''}
\end{quote}
This gives rise to the recursive nature of the problem formulation.
A breakdown of the variables used in the analysis, and their meaning, is provided in Table \ref{dp_vars}. \\

\begin{table}[htbp!]
\begin{centering}
    \begin{tabular}{|l|l|}
    \hline
    Variable & Meaning           \\ \hline
    $x_{t}$        & Decision Variable \\ \hline
    $s_{t}$        & State Variable    \\ \hline
    $t$        & Stage Number             \\ \hline
    $f$        & Return            \\ \hline
    $f*$       & Maximum Return    \\ \hline
    \end{tabular}
    \label{dp_vars}
    \caption{Variables used in the Dynamic Programming formulation.}
\end{centering}
\end{table}

A decision (the value of $x_{t}$) is required at each stage, where $t$ is the stage, and 1 $\le t \le$ 6.
Each stage has a state variable, $s_{t}$, and this state is divided into five states, one for each allowed percentage of the total budget to be bet.
At each stage (i.e. each round in the tournament), the dynamic programming system moves from the current state in the current stage to another ideal feasible state in the next stage.
This discrete dynamic programming methodology is able to solve each stage problem individually by considering the return function for the current stage, and the optimal function, $f*$, from the next stage ($t + 1$) to the end of the horizon (6 rounds).
In order to accomplish this, a recursive equation was constructed, as shown in Equation \ref{dp_recursive}.

\begin{equation}
f_{t}(s_{t}(i)) = \max \ [r_{t}(x_{t},s_{t}(i)) + f^{*}_{t+1}(s_{t+1}(j))]
\label{dp_recursive}
\end{equation}

This way, the system is able to determine optimal decisions for the problem overall, from any given state.
The full list of decisions and their individual returns are available at the end of the procedure.

Upon formulation of the problem, the dynamic programming portion of the system was then implemented in Matlab to generate solutions, given inputs from the prior mean-variance method.
The full implementation is given in Appendix A.
The data from the mean-variance solution was structured into a 5x5x6 3-dimensional matrix, where each row represents the returns of each state transition to the next stage.
Each ``page'' of this matrix (i.e. along the 3rd dimension) represents the data for each stage, or round, of the tournament.
With the data modelled appropriately, the optimal return (and its location) for the final round of the tournament was found and placed into a list holding each of the optimal decisions throughout the tournament.
Then, each stage of the tournament was looped through in a backwards fashion, and the optimal path of bets through the tournament was generated recursively.
The returns from each stage are then summed, to provide a look at the total optimal return for the tournament.

\section{Results and Analysis}
\subsection{Sample Inputs, Outputs, and Detailed Solution}


\subsection{Sensitivity Analysis}


\subsection{Inferences on Problem Solutions}

\newpage
\section{Conclusions}

\newpage
\section{References}
\bibliographystyle{IEEEtran}
\bibliography{bib}

\newpage
\section{Appendix A: Full Implementation}
\subsection{Dynamic Programming}
\begin{verbatim}
clear;

num_stages = 6;	% number of rounds in the tournament
num_states = 5;	% number of possible bets at each round

opt_list = [];	% stores optimal return and location of bet
[state_benefit, opt_dsg] = max(benefits(:, :, num_stages));
[row, col] = max(state_benefit);
state_benefit = row;
opt_dsg = opt_dsg(col);

opt_list = [opt_list; state_benefit, opt_dsg];

for curr_stage = (num_stages-1):-1:1
	[state_benefit, opt_dsg] = max(benefits(:, opt_dsg, curr_stage));
	opt_list = [opt_list; state_benefit, opt_dsg];
end

opt_list
\end{verbatim}


\end{document}